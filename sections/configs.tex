\chapter{Basic Functionalities}\label{chap:basics}

This chapter illustrates how to employ the most basic features of \columnflow.
By the end of it, you should be able to perform a calibration, apply a selection, calculate an observable and finally also produce the corresponding distribution for multiple processes.
Please note that this chapter is merely meant to summarize the most important aspects of these features.
For a more in-depth discussion and presentation, please consider Ref.~\cite{cf_repo}.

\section{Configuring the workflow}\label{sec:configs}

% TODO: write this section
Concepts to (briefly) introduce here
\begin{itemize}
	\item Order objects: Analysis, configs
	\item law config for module resolution
	\item brief walk through through demo config?
\end{itemize}

Might want to move this to Chapter 1.