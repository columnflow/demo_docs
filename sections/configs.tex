\section{Configuring the workflow}\label{sec:configs}

%Concepts to (briefly) introduce here
%\begin{itemize}
%	\item Order objects: Analysis, configs
%	\item law config for module resolution
%	\item brief walk through through demo config?
%\end{itemize}
%
%Might want to move this to Chapter 1.

This chapter gives an overview of the different modules that are needed to configure the workflow in general.
These modules can be divided into two groups.
On the one hand, there are modules to configure analysis-unspecific information, consisting of a metadata database containing general information about the data to process and the configuration for the \code{law} back-end.
On the other hand, analysis-specific information is needed.
This comprises of the specific list of physics processes and associated datasets that are needed to perform the analysis, as well as any additional information.
These groups are briefly described in the following.
For more information, please consider reading the corresponding documentation in Refs.~\cite{cf_repo,law,cmsdb,order}.

\subsection{Configuration of external information}

General information that is not specific to any given analysis is generally outsourced to other modules.
As already shown in Fig.~\ref{fig:directory_structure}, there are two git submodules to handle these aspects.

First, any analysis requires a pythonic interface to access information about the data to process.
Such a metadata database is realized with the \code{cmsdb}~\cite{cmsdb} project, which is based on \code{order} package~\cite{order}\footnote{For CMS analyses, this might be superseded by a centralized interface in the near future}.
This database organizes the datasets according to eras of data-taking and -processing.
Datasets need to have and identifier, or key, with which they can be accessed.
In the scope of this example, we will use the CMS data aggregation service (DAS) and its corresponding keys.
Additionally, datasets are generally attributed to different physics processes, which themselves have additional information such as relations to other processes and cross section predictions for different center-of-mass energies.
As the name suggests, \code{cmsdb} is tailored to the structure within the CMS collaboration.
However, a similar interface based on \code{order} can be created for any project. 

\begin{exercise}{Familiarize yourself with the metadata database}
	Have a look at the definitions in the \code{modules/cmsdb/cmsdb} directory.
	The campaign \code{run2\_2017\_nano\_v9} is of particular interest for this demonstration - have a look at the information that is compiled for the different datasets and physics processes.
\end{exercise}

\columnflow relies on \code{law}~\cite{law} to actually run the workflow.
This back-end is configured in the \code{law.cfg} file.




\subsection{Analysis-specific configuration}
bla


