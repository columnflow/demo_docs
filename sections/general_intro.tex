\chapter{Introduction to \columnflow}
\columnflow is intended as a back-end for analyses in order to facilitate processing large amounts of data.
It is purely python-based and employs multiple packages that are well-received and {-maintained} in the HEP community.
At the time of writing these instructions, the team of developer's purely consists of data analysts at the CMS experiment.
Therefore, this exercise is structured accordingly.
Please note that \columnflow is in principle designed in an experiment-agnostic way, such that it can also be extended to other use cases.

Additionally, please note that this hands-on exercise is not meant to fully document all available functionalities.
The purpose of this exercise is to give an overview of the most fundamental aspects and concepts that are available at the time of writing.
For a more comprehensive overview, please visit the official documentation~\cite{cf_repo}. % might want to put this as a proper reference
In case of any questions are comments, feel free to contact the maintainers for example via the git repository~\cite{cf_repo}.

\section{General Structure}
\begin{figure}[p]
	\centering
	\includegraphics[width=\textwidth]{images/CF_tasks.png}
	\Caption{\columnflow task graph hierarchy}{The tasks are arranged in three sections that correspond to general work packages when analysing data.
		The line strengths and styles indicate the behaviour when propagating information between tasks.
		For more information, please consider ref.~\cite{cf_repo}.
}
	\label{fig:task_graph}
\end{figure}

The guiding principle of \columnflow is that all analyses share basic work packages that need to be done when processing data.
Examples for such packages could be the calibration of relevant objects, applying selections to define a fiducial phase space for the analysis or the calculation of some sensitive observables, which are discussed in more detail in later chapters of this document.
\columnflow defines these work packages as law tasks, which can define dependencies amongst each other and will only run necessary tasks to obtain the requested output.


Figure~\ref{fig:task_graph} depicts an overview of the available tasks and their dependencies.
The highlighted regions indicate use cases that are discussed in chapter~\ref{chap:basics}.