\section{Writing a Selector}\label{sec:selector}

The \CCSPStlye{Selector} class should be used to implement analysis selections.
This is a crucial step in the workflow since the decision to keep or reject objects or even whole events is performed here.
Since the selection usually depends on for example four momenta of the objects within the events, it is executed after the calibration.
The corresponding task is called \CCSPStlye{cf.SelectEvents}.
For more information, please consider Ref.~\cite{cf_repo}.

\begin{table}[t]
	\Caption{Selection criteria for leptons}{Shown are the selection criteria for electrons (muons) at the 'loose' and 'tight'}
\end{table}
In this part of the tutorial, we will write selections for electrons and muons.
\textbf{\underline{Loose Electrons}}
\begin{itemize}
    \item
\end{itemize}
