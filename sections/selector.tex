\section{Writing a Selector}\label{sec:selector}

The \CCSPStlye{Selector} class should be used to implement analysis selections.
This is a crucial step in the workflow since the decision to keep or reject objects or even whole events is performed here.
Since the selection usually depends on for example four-momenta of the objects within the events, it is executed after the calibration.
The corresponding task is called \CCSPStlye{cf.SelectEvents}.

For more information, please consider Ref.~\cite{cf_repo}.

\renewcommand{\arraystretch}{1.5}
\begin{table}[h!]
    \centering
    \begin{tabular}{|m{4cm}|m{5cm}|m{5.5cm}|}
    \hline
    & \textbf{Electrons} & \textbf{Muons} \\ \hline
    \textbf{Kinematic cuts} &
    \begin{itemize}[leftmargin=*]
    \item $p_T^e > 7$ GeV
    \item $|\eta^e| < 2.5$
    \end{itemize} &
    \begin{itemize}[leftmargin=*]
        \item $p_T^\mu > 5$ GeV
        \item $|\eta^\mu| < 2.5$
    \end{itemize} \\ \hline
    \textbf{Vertex cuts} &
    \begin{itemize}[leftmargin=*]
        \item $d_{xy} < 0.5$
        \item $d_z < 1$ cm
        \item $SIP < 4$
    \end{itemize} &
    \begin{itemize}[leftmargin=*]
        \item $d_{xy} < 0.5$
        \item $d_z < 1$ cm
        \item $SIP < 4$
    \end{itemize} \\ \hline
    \textbf{Isolation \& ID for \newline 'tight' working point} & Dedicated BDT targeting \newline prompt electrons. & Select only muons within \newline a well-defined cone ($R=0.35$). \\ \hline
    \end{tabular}
    \Caption{Selection criteria for leptons.}{Shown are the selection cuts for electrons/muons at the 'loose' working point, with the last row defining the extra requirement for the leptons to pass the 'tight' working point.}
    \label{leptonSelection}
\end{table}

In this part of the tutorial, we will write selections for electrons and muons.
In the script \code{h4l/selection/lepton.py} you can find the base structure to implement two \CCSPStlye{Selector} modules, \code{electron\_selection} and \code{muon\_selection}.
Each of these objects uses the relevant event information for its implementation.


\textbf{\underline{Electron Selection}}

For \code{electron\_selection}, the electron kinematic information is first loaded into the \code{uses} set.
Then, information that is dependent on the nanoAOD version is loaded, in this case which MVA (Multi-Variate Analysis) flag should be used.
Notice that we use the union operator \code{|} to append either \code{Electron.mvaFall17V2Iso} or \code{Electron.mvaHZZIso} to the set containing the kinematic variables.
Lastly, to perform four-vector calculations, we also require \code{attach\_coffea\_behavior}, which is imported at the beginning of the script from \code{columnflow.selection.util}.


This \CCSPStlye{Selector} object also has two more dependencies, \code{exposed} and \code{sandbox}.
The first one determines whether or not the \CCSPStlye{Selector} object is available from the command line.
As for \code{sandbox}, it specifies the software enviorment where this \CCSPStlye{Selector} will be executed. A detailed description can be found in \href{https://columnflow.readthedocs.io/en/latest/user_guide/sandbox.html}{Sandboxes and Their Use in Columnflow}.

\newpage
\begin{itemize}
    \item {
        \textbf{\underline{Loose Electrons}} -- Within the main body of \code{electron\_selection}, all selections should be applied.
        Note that the minimum transverse momentum has already been specified in \code{min\_pt}.
        The actual value, in GeV, is set in \code{h4l/config/config\_h4l.py}, and depends on the argument \code{working\_point}.
        In the config file, a dictionary stores two possible values, $15$ for a \code{'tight'} working point (default value), and $7$ for a \code{'loose'} working point.
        Both the transverse momentum and the pseudorapidity selection criteria have already been applied in \code{default\_mask}.
        You should now complete the mask with the remaining selection criteria from Table \ref{leptonSelection}.
    }
    \item {
        \textbf{\underline{Tight Electrons}} -- Finally, you should also add a condition that applies the identification criteria for when \code{working\_point} is set to \code{'tight'}.
        Both the fSCeta and BDT values are set in the function \code{return\_electron\_id\_cuts}, which can be found in \code{h4l/selection/util.py}.
    }
\end{itemize}

After all selections have been applied, the final part of the module sorts all events by their momentum and applies the \code{default\_mask}.
The indices of selected events are then stored in \code{selected\_electron\_idx}.
The \code{electron\_selection} module finally returns both all events and a \CCSPStlye{SelectionResult} class instance.
We initiate the \CCSPStlye{SelectionResult} instance by setting the \code{objects} and \code{aux} (i.e. auxiliary) arguments.
Within \code{objects}, a nested dictionary saves \code{selected\_electron\_idx} as a value to an \code{Electron} key.
The selection mask itself, \code{default\_mask} is stored in \code{aux}.

\begin{exercise}{Writing a Selector -- Electron Selection}[h4l/selection/lepton\_solution.py]
	Refering to Table \ref{leptonSelection}, fill in the missing information in the \CCSPStlye{Selector} module \code{electron\_selection} defined in \code{h4l/selection/lepton.py}.
\end{exercise}

\vspace{0.8cm}

\textbf{\underline{Muon Selection}}

The \code{muon\_selection} module behaves very similarly. In this case, a dedicated software environment is not required.
There is also no information dependent on the nanoAOD version.
Besides the kinematic information, the \code{uses} set also loads muon quality criteria (e.g. if it is a global or tracker muon), identification and isolation information.

\begin{itemize}
    \item {
        \textbf{\underline{Loose Muons}} -- Within the main body of \code{muon\_selection}, a selection mask is now defined \code{selected\_muon\_mask}.
        Similarly to the electron selection, the minimum transverse momentum and pseudorapidity are already defined.
        You should now expand this mask such that:
        \begin{enumerate}
            \item you require either a global or tracker muon (for tracker muons \code{nStations} should be a positive);
            \item discard standalone muons if the reconstructed tracks are only present in the muon system (i.e. for standalone muons, you should require a positive number of \code{nTrackerLayers});
            \item apply the remaining selection criteria from Table \ref{leptonSelection}.
        \end{enumerate}
    }
    \item {
        \textbf{\underline{Tight Muons}} -- You should now add three conditions to \code{selected\_muon\_mask}:
        \begin{enumerate}
            \item enforce that the low momentum muons ($< 200$ GeV) are ParticleFlow candidates (use the variable \code{isPFcand});
            \item enforce that the high momentum muons ($\geq 200$ GeV) are ParticleFlow candidates OR have a positive \code{highPtId};
            \item use the variable \code{pfRelIso03\_all} to apply the condition in Table \ref{leptonSelection}.
        \end{enumerate}
    }
\end{itemize}

\begin{exercise}{Writing a Selector -- Muon Selection}[h4l/selection/lepton\_solution.py]
	Again refering to Table \ref{leptonSelection}, fill in the missing information in the \CCSPStlye{Selector} module \code{muon\_selection} defined in \code{h4l/selection/lepton.py}.
\end{exercise}
