\section{Writing a Calibrator}\label{sec:calibrator}

\CCSPStlye{Calibrator}s are dedicated \CCSPStlye{TaskArrayFunctions} that perform a calibration of objects, such as jets, leptons or missing transverse energy.
Since this calibration modifies the four-momenta in the events, they can influence the selection of a given analysis.
Therefore, calibrations should generally be performed before applying analysis selections.
The associated task within the workflow is \CCSPStlye{cf.CalibrateEvents}, which is executed before the selection modules within the task graph.

\columnflow provides generally-used calibrations for different objects which follow the common (CMS) guidelines.
For more information about which calibrations are implemented and how to use them, we recommend to consult the current status of the documentation~\cite{cf_repo}.

The \code{H4L} analysis includes an exemplary \CCSPStlye{Calibrator} in \code{h4l/calibration/jets.py}.
In this module, we perform a calibration of jets in our events, which is based on the implementation that comes with \columnflow itself.

% TODO: include code here?
%\begin{lstlisting}[language=python]
%	# coding: utf-8
%	
%	"""
%	Jet energy calibration methods.
%	"""
%	
%	from columnflow.calibration import Calibrator, calibrator
%	from columnflow.calibration.cms.jets import jec, jer
%	from columnflow.util import maybe_import
%	
%	ak = maybe_import("awkward")
%	
%	
%	# custom jec calibrator that only runs nominal correction
%	jec_nominal = jec.derive("jec_nominal", cls_dict={"uncertainty_sources": []})
%	
%	
%	@calibrator(
%	uses={jec_nominal},
%	produces={jec_nominal},
%	)
%	def jet_energy(self: Calibrator, events: ak.Array, **kwargs) -> ak.Array:
%	"""
%	Common calibrator for jet energy corrections, applying nominal JEC for data, and JEC with
%	uncertainties plus JER for MC. Information about used and produced columns and dependent
%	calibrators is added in a custom init function below.
%	"""
%	# correct jet energy scale
%	events = self[jec_nominal](events, **kwargs)
%	
%	# jet energy resolution smearing (MC only)
%	if self.dataset_inst.is_mc:
%	events = self[jer](events, **kwargs)
%	
%	return events
%	
%	
%	@jet_energy.init
%	def jet_energy_init(self: Calibrator) -> None:
%	# return immediately if dataset object has not been loaded yet
%	if not getattr(self, "dataset_inst", None):
%	return
%	
%	# add columns producs by JER smearing calibrator (MC only)
%	if self.dataset_inst.is_mc:
%	self.uses.add(jer)
%	self.produces.add(jer)
%	
%\end{lstlisting}

First the relevant modules are imported.
Note that \code{awkward} is loaded with the \code{maybe\_import} mechanism.
This is necessary due to the encapsulated structure of the underlying software stack.
In the scope of this exercise, we don't want to consider all the different sources of uncertainties that are associated with jet calibration yet.
Therefore, we use the \code{derive} mechanism of \CCSPStlye{TaskArrayFunctions} to define a new class called \code{jec\_nominal}, which inherits from the original \code{jec} \CCSPStlye{Calibrator} but overwrites the corresponding class member variable.

Next, we define our new \CCSPStlye{Calibrator} class \code{jet\_energy} as shown before.
Since we want to call the \code{jec\_nominal} class within this \CCSPStlye{Calibrator}, we need to add it to the \code{uses} set.
This will load all columns that \code{jec\_nominal} needs, and will additionally make \code{jec\_nominal} accessible within the main body of our new \CCSPStlye{Calibrator} as shown below.
We also want to save all columns that \code{jec\_nominal} produces to disk for later use, which is why we need to add it to the \code{produces} set as well.

All \CCSPStlye{TaskArrayFunction}s have access to information of the current point within the task graph, such as the \code{config} object mentioned in Sec.~\ref{sec:configs} and the current dataset that is processed.
Their behavior can depend on this information, which is shown for the jet energy resolution~(JER) calibration of our new \code{jet\_energy} module.
JER needs to be applied to simulated samples only, which is realized in the code correspondingly.
The set of columns to be loaded from disk is also dynamically configured in the \code{init} function of the \code{jet\_energy} \CCSPStlye{Calibrator} such that columns corresponding to JER are only added to the \code{uses} and \code{produces} sets if necessary.

\begin{exercise}{Writing a Calibrator}%[h4l/calibration/jet.py]
	Familiarize yourself with how the \code{jet\_energy} \CCSPStlye{Calibrator} works.
\end{exercise}