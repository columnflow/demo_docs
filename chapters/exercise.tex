\chapter{Basic Functionalities}\label{chap:basics}

This chapter illustrates how to employ the most basic features of \columnflow. By the end of it, you should be able to perform a calibration, apply a selection, calculate an observable and finally also produce the corresponding distribution for multiple processes. Please note that this chapter is merely meant to summarize the most important aspects of these features. For a more in-depth discussion and presentation, please consider Ref.~\cite{cf_repo}.

\chapter{Basic Functionalities}\label{chap:basics}

This chapter illustrates how to employ the most basic features of \columnflow.
By the end of it, you should be able to perform a calibration, apply a selection, calculate an observable and finally also produce the corresponding distribution for multiple processes.
Please note that this chapter is merely meant to summarize the most important aspects of these features.
For a more in-depth discussion and presentation, please consider Ref.~\cite{cf_repo}.

\section{Configuring the workflow}\label{sec:configs}

% TODO: write this section
Concepts to (briefly) introduce here
\begin{itemize}
	\item Order objects: Analysis, configs
	\item law config for module resolution
	\item brief walk through through demo config?
\end{itemize}

Might want to move this to Chapter 1.


\section{The mother of all: TaskArrayFunctions}\label{sec:taskarrayfunc}

Before getting started with the details of the implementation, we will cover the basic structure of the most relevant building blocks of \columnflow.
These objects are all derived from the so-called \CCSPStlye{TaskArrayFunction}, which defines hooks and interfaces to propagate information from your objects to the \columnflow tasks.

This class of objects can for example explicitly define some runtime dependencies with the following member variables:
\begin{description}
	\item[uses] is a set of column names that are to be retrieved from disk.
	You can also provide other \CCSPStlye{TaskArrayFunction}s here, in which case their \texttt{uses} set is appended to this one.
	\item[produces] is a set of columns that are to be written to disk.
	You can also provide other \CCSPStlye{TaskArrayFunction}s here, in which case their \texttt{produces} set is appended to this one.
	\item[sandbox] is a hook that is propagated to the actual Task instance that is run and calls your module.
	This defines the software environment in which your module needs to run, which allows for a granular definition of the required software and can minimize the overhead of your software packages.
	
\end{description}

For convenience, all  \CCSPStlye{TaskArrayFunction}s provide decorators to easily define new modules:

\begin{lstlisting}[language=python]
	# assuming you want to define the TaskArrayFunction example
	@example(
		# for example, request the Jet pt, all Electron information and everything another 
		uses = {
			"Jet.pt",  # request transverse momentum for all jets
			"Electron.*",  # request all information for electrons
			some_other_TaskArrayFunction  # propagate everything some_other_TaskArrayFunction needs to this example TaskArrayFunction
	    },
	    # define which outputs are to be written to disk
	    produces={
	    	"some_fancy_output"
	    },
	    # define in what kind of software environment this module should be run
	    sandbox="some_cool_sandbox"
	)
	def your_new_example_module(events: ak.Array, **kwargs):
	    # this is the main body of your module, do something here...
\end{lstlisting}

Additionally, \CCSPStlye{TaskArrayFunction}s provide a set of hooks, three of which are of special importance and are briefly introduced in the following:
\begin{description}
	\item[init] defines instructions that are to be done when this object is first initialised.
	\item[requires] adds object-specific requirements on top of the pre-existing Task-level requirements.
	This allows to explicitly define dependencies and can for example ensure that the output of another module is calculated before starting with the current task.
	\item[setup] is run before actually entering the main body of your module that performs e.g.\ calculations.
	This is for example useful to parse the output of the aforementioned requirements such that your object can also use it.
	
\end{description}

These hooks can be added to an existing  \CCSPStlye{TaskArrayFunction} instance with dedicated decorators like so:

\begin{lstlisting}[language=python]
	# assuming you have defined your_new_example_module from the example above
	
	# define your init function
	@your_new_example_module.init
	def some_init_func_name(self):
	    # do something when your_new_example_module is first initialized
	    
	# define some special requirements for your module
	@your_new_example_module.requires
	def some_func_name_for_requires(self):
	    # add some requirements
	
	# do something before entering the main body of your module
	@your_new_example_module.setup
	def some_setup_func_name(self, **kwargs):
	    # prepare your module so it runs smoothly
\end{lstlisting}

In the following, these concepts will be shown with concrete details.

\section{Writing a Calibrator}\label{sec:calibrator}
\section{Writing a Selector}\label{sec:selector}

The \CCSPStlye{Selector} class should be used to implement analysis selections.
This is a crucial step in the workflow since the decision to keep or reject objects or even whole events is performed here.
Since the selection usually depends on for example four-momenta of the objects within the events, it is executed after the calibration.
The corresponding task is called \CCSPStlye{cf.SelectEvents}.

For more information, please consider Ref.~\cite{cf_repo}.

\renewcommand{\arraystretch}{1.5}
\begin{table}[h!]
    \centering
    \begin{tabular}{|m{4cm}|m{5cm}|m{5.5cm}|}
    \hline
    & \textbf{Electrons} & \textbf{Muons} \\ \hline
    \textbf{Kinematic cuts} &
    \begin{itemize}[leftmargin=*]
    \item $p_T^e > 7$ GeV
    \item $|\eta^e| < 2.5$
    \end{itemize} &
    \begin{itemize}[leftmargin=*]
        \item $p_T^\mu > 5$ GeV
        \item $|\eta^\mu| < 2.5$
    \end{itemize} \\ \hline
    \textbf{Vertex cuts} &
    \begin{itemize}[leftmargin=*]
        \item $d_{xy} < 0.5$
        \item $d_z < 1$ cm
        \item $SIP < 4$
    \end{itemize} &
    \begin{itemize}[leftmargin=*]
        \item $d_{xy} < 0.5$
        \item $d_z < 1$ cm
        \item $SIP < 4$
    \end{itemize} \\ \hline
    \textbf{Isolation \& ID for \newline 'tight' working point} & Dedicated BDT targeting \newline prompt electrons. & Select only muons within \newline a well-defined cone ($R=0.35$). \\ \hline
    \end{tabular}
    \Caption{Selection criteria for leptons.}{Shown are the selection cuts for electrons/muons at the 'loose' working point, with the last row defining the extra requirement for the leptons to pass the 'tight' working point.}
    \label{leptonSelection}
\end{table}

In this part of the tutorial, we will write selections for electrons and muons.
In the script \code{h4l/selection/lepton.py} you can find the base structure to implement two \CCSPStlye{Selector} modules, \code{electron\_selection} and \code{muon\_selection}.
Each of these objects uses the relevant event information for its implementation.


\textbf{\underline{Electron Selection}}

For \code{electron\_selection}, the electron kinematic information is first loaded into the \code{uses} set.
Then, information that is dependent on the nanoAOD version is loaded. In this case, which MVA (Multi-Variate Analysis) flag should be used.
Notice that we use the union operator \code{|} to append either \code{Electron.mvaFall17V2Iso} or \code{Electron.mvaHZZIso} to the set containing the kinematic variables.
Lastly, to perform four-vector calculations, we also recquire \code{attach\_coffea\_behavior}, which is imported at the beginning of the script from \code{columnflow.selection.util}.


This \CCSPStlye{Selector} object also has two more dependencies, \code{exposed} and \code{sandbox}.
The first one determines whether or not the \CCSPStlye{Selector} object is available from the command line.
As mentioned in the last section, \code{sandbox} specifies the software enviorment where this \CCSPStlye{Selector} will be executed.

\newpage
\begin{itemize}
    \item {
        \textbf{\underline{Loose Electrons}} -- Within the main body of \code{electron\_selection}, all selections should be applied.
        Note that the minimum transverse momentum has already been specified in \code{min\_pt}.
        The actual value, in GeV, is set in \code{h4l/config/config\_h4l.py}, and depends on the argument \code{working\_point}.
        In the config file, a dictionary stores two possible values, $15$ for a \code{'tight'} working point (default value), and $7$ for a \code{'loose'} working point.
        Both the transverse momentum and the pseudorapidity selection criteria have already been applied in \code{default\_mask}.
        You should now complete the mask with the remaining selection criteria from Table \ref{leptonSelection}.
    }
    \item {
        \textbf{\underline{Tight Electrons}} -- Finally, you should also add a condition that applies the identification criteria for when \code{working\_point} is set to \code{'tight'}.
        Both the fSCeta and BDT values are set in the function \code{return\_electron\_id\_cuts}, which can be found in \code{h4l/selection/util.py}.
    }
\end{itemize}

After all selections have been applied, the final part of the module sorts all events by their momentum and applies the \code{default\_mask}.
The indices of selected events are then stored in \code{selected\_electron\_idx}.
The \code{electron\_selection} module finally returns both all events and a \CCSPStlye{SelectionResult} class instance.
We initiate the \CCSPStlye{SelectionResult} instance by setting the \code{objects} and \code{aux} (i.e. auxiliary) arguments.
Within \code{objects}, a nested dictionary saves \code{selected\_electron\_idx} as a value to an \code{Electron} key.
The selection mask itself, \code{default\_mask} is stored in \code{aux}.

\begin{exercise}{Writing a Selector -- Electron Selection}[h4l/selection/lepton\_solution.py]
	Refering to Table \ref{leptonSelection}, fill in the missing information in the \CCSPStlye{Selector} module \code{electron\_selection} defined in \code{h4l/selection/lepton.py}.
\end{exercise}

\vspace{0.8cm}

\textbf{\underline{Muon Selection}}

The \code{muon\_selection} module behaves very similarly. In this case, a dedicated software environment is not required.
There is also no information dependent on the nanoAOD version.
Besides the kinematic information, the \code{uses} set also loads muon quality criteria (e.g. if it is a global or tracker muon), identification and isolation information.

\begin{itemize}
    \item {
        \textbf{\underline{Loose Muons}} -- Within the main body of \code{muon\_selection}, a selection mask is now defined \code{selected\_muon\_mask}.
        Similarly to the electron selection, the minimum transverse momentum and pseudorapidity are already defined.
        You should now expand this mask such that:
        \begin{enumerate}
            \item you recquire either a global or tracker muon (for tracker muons \code{nStations} should be a positive);
            \item discard standalone muon if the reconstructed tracks are only present in the muon system (i.e. for standalone muons, you should recquire a positive number of \code{nTrackerLayers});
            \item apply the remaining selection criteria from Table \ref{leptonSelection}.
        \end{enumerate}
    }
    \item {
        \textbf{\underline{Tight Muons}} -- You should now add three conditions to \code{selected\_muon\_mask}:
        \begin{enumerate}
            \item enforce that the low momentum muons ($< 200$ GeV) are ParticleFlow candidates (use the variable \code{isPFcand});
            \item enforce that the high momentum muons ($\geq 200$ GeV) are ParticleFlow candidates OR have a positive \code{highPtId};
            \item use the variable \code{pfRelIso03\_all} to apply the condition in Table \ref{leptonSelection}.
        \end{enumerate}
    }
\end{itemize}

\begin{exercise}{Writing a Selector -- Muon Selection}[h4l/selection/lepton\_solution.py]
	Again refering to Table \ref{leptonSelection}, fill in the missing information in the \CCSPStlye{Selector} module \code{muon\_selection} defined in \code{h4l/selection/lepton.py}.
\end{exercise}

\section{Writing a Producer}\label{sec:producer}

The \CCSPStlye{Producer} class is used to calculate higher-level variables and define new array columns to be written to disk.
The corresponding task is called \CCSPStlye{cf.ProduceColumns} (see Ref.~\cite{cf_repo} for detailed information).
Naturally, we only want to compute these new variables for the relevant events for our analysis.
Thus, the producers are executed after the selection step in the task graph. 

ColumnFlow also provides \CCSPStlye{Producer}s to compute commonly used event information, such as MC, pdf or pileup weights. These can be found under \code{columnflow.production.cms}. The \code{H4L} analysis includes three exemplary \CCSPStlye{Producer}s in \code{h4l/production/example.py}. You will notice that the script starts by importing all relevant modules, including CMS specific ones. We also load both \code{numpy} and \code{awkward} with the \code{maybe\_import} mechanism.

We start by defining a new \CCSPStlye{Producer} class named \code{features}. This class requires the transverse jet momentum \code{Jet.pt}, which must be added to its \code{uses} set, and produces two new array columns, the total jet transverse momentum \code{ht} and the number of jets in an event \code{n\_jet}, which are both added to its \code{produces} set. Each of these new variables is computed and stored to disk using the \code{set\_ak\_column} function. Note that for the case of \code{n\_jet}, we specified that the column element must be an \code{int} value.

The second \CCSPStlye{Producer} class \code{cutflow\_features} allows us to define and store features to be used for cutflow plots. Here, in addition to \code{Jet.pt} we also require \code{mc\_weight} and \code{category\_ids} to be added to the \code{uses} set. Note that both of these are \CCSPStlye{Producer}s themselves which you can find by following the import path at the beginning of the script. 

The  \CCSPStlye{Producer} class \code{mc\_weight} reads in the \code{genWeight} column and, if existent, the \code{LHEWeight} column, both stored in \code{events}. Since these columns are required, they are both added to the \code{uses} set of \code{mc\_weight}. By extension, when we call \code{mc\_weight} in our \code{uses} set, we are calling these columns as well. The \code{mc\_weight} class simply decides which one of these weights to use and saves the decision as a new column, also named \code{mc\_weight}, which is included in its \code{produces} set. Again, by extension, we must add the \code{mc\_weight} class to our own \code{produces} set such that this new column also gets created and saved to disk. 

Meanwhile, the \code{category\_ids} class assigns each event an array of category ids, which it stores as a new column also named \code{category\_ids}. Thus, we must also add this class to our \code{produces} set. The topic of defining categories is discussed in detail in the Section \ref{sec:categories}. Now that we have access to both these \CCSPStlye{Producer} classes, the \code{cutflow\_features} class can use them to attribute MC weights (if the dataset passed to it is tagged as an MC dataset) and category ids to \code{events}. It then creates a new column in the updated \code{events} object named \code{cutflow.jet1\_pt} which saves the transverse momentum of the most energetic jet in each event stored in \code{Jet.pt\text{[:,0]}}. If the event does not contain jets, it instead saves an \code{EMPTY\_FLOAT} value. 

The last \CCSPStlye{Producer} class defined is \code{example} and follows the same structure as the  two previously explained \CCSPStlye{Producer}s. First, it starts by creating the \code{cutflow.jet1\_pt} column by using the \CCSPStlye{Producer} class \code{features} called at \code{\text{events=self[features](events, **kwargs)}}. It then applies category ids and determinist seeds to the updated \code{events} object. Lastly, it applies normalization weights and muon weights only to \code{events} tagged as originating from a MC dataset. 

\begin{exercise}{Understanding some basic Producers}
	Familiarize yourself with the \CCSPStlye{Producer} classes mentioned above.
\end{exercise}

In this \code{H4L} analysis we want to calculate the four-lepton invariant mass, so that we can see the resonant Higgs peak. Note that to perform four-vector calculations, you need to import \code{attach\_coffea\_behavior} from \code{columnflow.production.util}. You will need to use kinematic information from both the \code{events.Electron} and \code{events.Muon} collections and create a new column which stores your calculated invariant mass.

\begin{exercise}{Writing a Producer}{}
	Write a \CCSPStlye{Producer} class which computes the four lepton invariant mass. 
\end{exercise}
 

