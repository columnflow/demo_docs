\chapter{Basic Functionalities}\label{chap:basics}

This chapter illustrates how to employ the most basic features of \columnflow. By the end of it, you should be able to perform a calibration, apply a selection, calculate an observable and finally also produce the corresponding distribution for multiple processes. Please note that this chapter is merely meant to summarize the most important aspects of these features. For a more in-depth discussion and presentation, please consider Ref.~\cite{cf_repo}.

\section{Configuring the workflow}\label{sec:configs}

%Concepts to (briefly) introduce here
%\begin{itemize}
%	\item Order objects: Analysis, configs
%	\item law config for module resolution
%	\item brief walk through through demo config?
%\end{itemize}
%
%Might want to move this to Chapter 1.

This chapter gives an overview of the different modules that are needed to configure the workflow in general.
These modules can be divided into two groups.
On the one hand, there are modules to configure analysis-unspecific information, consisting of a metadata database containing general information about the data to process and the configuration for the \code{law} back-end.
On the other hand, analysis-specific information is needed.
This comprises of the specific list of physics processes and associated datasets that are needed to perform the analysis, as well as any additional information.
These groups are briefly described in the following.
For more information, please consider reading the corresponding documentation in Refs.~\cite{cf_repo,law,cmsdb,order}.

\subsection{Configuration of external information}

General information that is not specific to any given analysis is generally outsourced to other modules.
As already shown in Fig.~\ref{fig:directory_structure}, there are two git submodules to handle these aspects.

First, any analysis requires a pythonic interface to access information about the data to process.
Such a metadata database is realized with the \code{cmsdb}~\cite{cmsdb} project, which is based on \code{order} package~\cite{order}\footnote{For CMS analyses, this might be superseded by a centralized interface in the near future}.
This database organizes the datasets according to eras of data-taking and -processing.
Datasets need to have and identifier, or key, with which they can be accessed.
In the scope of this example, we will use the CMS data aggregation service (DAS) and its corresponding keys.
Additionally, datasets are generally attributed to different physics processes, which themselves have additional information such as relations to other processes and cross section predictions for different center-of-mass energies.
As the name suggests, \code{cmsdb} is tailored to the structure within the CMS collaboration.
However, a similar interface based on \code{order} can be created for any project. 

\begin{exercise}{Familiarize yourself with the metadata database}
	Have a look at the definitions in the \code{modules/cmsdb/cmsdb} directory.
	The campaign \code{run2\_2017\_nano\_v9} is of particular interest for this demonstration - have a look at the information that is compiled for the different datasets and physics processes.
\end{exercise}

\columnflow relies on \code{law}~\cite{law} to actually run the workflow.
This back-end is configured in the \code{law.cfg} file.




\subsection{Analysis-specific configuration}
bla



\include{sections/taskarrayfunctions}
\section{Writing a Calibrator}
\label{calibrator}
!! json mirror for afs needs to be fixed in the podas config. new mirror in :

\texttt{/afs/cern.ch/work/m/mrieger/public/mirrors/jsonpog-integration-f4f0c907}

\section{Writing a Selector}\label{sec:selector}

The \CCSPStlye{Selector} class should be used to implement analysis selections.
This is a crucial step in the workflow since the decision to keep or reject objects or even whole events is performed here.
Since the selection usually depends on for example four-momenta of the objects within the events, it is executed after the calibration.
The corresponding task is called \CCSPStlye{cf.SelectEvents}.

For more information, please consider Ref.~\cite{cf_repo}.

\renewcommand{\arraystretch}{1.5}
\begin{table}[h!]
    \centering
    \begin{tabular}{|m{4cm}|m{5cm}|m{5.5cm}|}
    \hline
    & \textbf{Electrons} & \textbf{Muons} \\ \hline
    \textbf{Kinematic cuts} &
    \begin{itemize}[leftmargin=*]
    \item $p_T^e > 7$ GeV
    \item $|\eta^e| < 2.5$
    \end{itemize} &
    \begin{itemize}[leftmargin=*]
        \item $p_T^\mu > 5$ GeV
        \item $|\eta^\mu| < 2.5$
    \end{itemize} \\ \hline
    \textbf{Vertex cuts} &
    \begin{itemize}[leftmargin=*]
        \item $d_{xy} < 0.5$
        \item $d_z < 1$ cm
        \item $SIP < 4$
    \end{itemize} &
    \begin{itemize}[leftmargin=*]
        \item $d_{xy} < 0.5$
        \item $d_z < 1$ cm
        \item $SIP < 4$
    \end{itemize} \\ \hline
    \textbf{Isolation \& ID for \newline 'tight' working point} & Dedicated BDT targeting \newline prompt electrons. & Select only muons within \newline a well-defined cone ($R=0.35$). \\ \hline
    \end{tabular}
    \Caption{Selection criteria for leptons.}{Shown are the selection cuts for electrons/muons at the 'loose' working point, with the last row defining the extra requirement for the leptons to pass the 'tight' working point.}
    \label{leptonSelection}
\end{table}

In this part of the tutorial, we will write selections for electrons and muons.
In the script \code{h4l/selection/lepton.py} you can find the base structure to implement two \CCSPStlye{Selector} modules, \code{electron\_selection} and \code{muon\_selection}.
Each of these objects uses the relevant event information for its implementation.


\textbf{\underline{Electron Selection}}

For \code{electron\_selection}, the electron kinematic information is first loaded into the \code{uses} set.
Then, information that is dependent on the nanoAOD version is loaded, in this case which MVA (Multi-Variate Analysis) flag should be used.
Notice that we use the union operator \code{|} to append either \code{Electron.mvaFall17V2Iso} or \code{Electron.mvaHZZIso} to the set containing the kinematic variables.
Lastly, to perform four-vector calculations, we also recquire \code{attach\_coffea\_behavior}, which is imported at the beginning of the script from \code{columnflow.selection.util}.


This \CCSPStlye{Selector} object also has two more dependencies, \code{exposed} and \code{sandbox}.
The first one determines whether or not the \CCSPStlye{Selector} object is available from the command line.
As mentioned in the last section, \code{sandbox} specifies the software enviorment where this \CCSPStlye{Selector} will be executed.

\newpage
\begin{itemize}
    \item {
        \textbf{\underline{Loose Electrons}} -- Within the main body of \code{electron\_selection}, all selections should be applied.
        Note that the minimum transverse momentum has already been specified in \code{min\_pt}.
        The actual value, in GeV, is set in \code{h4l/config/config\_h4l.py}, and depends on the argument \code{working\_point}.
        In the config file, a dictionary stores two possible values, $15$ for a \code{'tight'} working point (default value), and $7$ for a \code{'loose'} working point.
        Both the transverse momentum and the pseudorapidity selection criteria have already been applied in \code{default\_mask}.
        You should now complete the mask with the remaining selection criteria from Table \ref{leptonSelection}.
    }
    \item {
        \textbf{\underline{Tight Electrons}} -- Finally, you should also add a condition that applies the identification criteria for when \code{working\_point} is set to \code{'tight'}.
        Both the fSCeta and BDT values are set in the function \code{return\_electron\_id\_cuts}, which can be found in \code{h4l/selection/util.py}.
    }
\end{itemize}

After all selections have been applied, the final part of the module sorts all events by their momentum and applies the \code{default\_mask}.
The indices of selected events are then stored in \code{selected\_electron\_idx}.
The \code{electron\_selection} module finally returns both all events and a \CCSPStlye{SelectionResult} class instance.
We initiate the \CCSPStlye{SelectionResult} instance by setting the \code{objects} and \code{aux} (i.e. auxiliary) arguments.
Within \code{objects}, a nested dictionary saves \code{selected\_electron\_idx} as a value to an \code{Electron} key.
The selection mask itself, \code{default\_mask} is stored in \code{aux}.

\begin{exercise}{Writing a Selector -- Electron Selection}[h4l/selection/lepton\_solution.py]
	Refering to Table \ref{leptonSelection}, fill in the missing information in the \CCSPStlye{Selector} module \code{electron\_selection} defined in \code{h4l/selection/lepton.py}.
\end{exercise}

\vspace{0.8cm}

\textbf{\underline{Muon Selection}}

The \code{muon\_selection} module behaves very similarly. In this case, a dedicated software environment is not required.
There is also no information dependent on the nanoAOD version.
Besides the kinematic information, the \code{uses} set also loads muon quality criteria (e.g. if it is a global or tracker muon), identification and isolation information.

\begin{itemize}
    \item {
        \textbf{\underline{Loose Muons}} -- Within the main body of \code{muon\_selection}, a selection mask is now defined \code{selected\_muon\_mask}.
        Similarly to the electron selection, the minimum transverse momentum and pseudorapidity are already defined.
        You should now expand this mask such that:
        \begin{enumerate}
            \item you recquire either a global or tracker muon (for tracker muons \code{nStations} should be a positive);
            \item discard standalone muon if the reconstructed tracks are only present in the muon system (i.e. for standalone muons, you should recquire a positive number of \code{nTrackerLayers});
            \item apply the remaining selection criteria from Table \ref{leptonSelection}.
        \end{enumerate}
    }
    \item {
        \textbf{\underline{Tight Muons}} -- You should now add three conditions to \code{selected\_muon\_mask}:
        \begin{enumerate}
            \item enforce that the low momentum muons ($< 200$ GeV) are ParticleFlow candidates (use the variable \code{isPFcand});
            \item enforce that the high momentum muons ($\geq 200$ GeV) are ParticleFlow candidates OR have a positive \code{highPtId};
            \item use the variable \code{pfRelIso03\_all} to apply the condition in Table \ref{leptonSelection}.
        \end{enumerate}
    }
\end{itemize}

\begin{exercise}{Writing a Selector -- Muon Selection}[h4l/selection/lepton\_solution.py]
	Again refering to Table \ref{leptonSelection}, fill in the missing information in the \CCSPStlye{Selector} module \code{muon\_selection} defined in \code{h4l/selection/lepton.py}.
\end{exercise}

\section{Writing a Producer}\label{sec:producer}

The \CCSPStlye{Producer} class is used to calculate higher-level observables and define new columns to be written o disk. The corresponding task is called \CCSPStlye{cf.ProduceColumns} (see Ref.~\cite{cf_repo} for detailed info). Naturally, we only want to compute these new variables for the relevant events for our analysis. Thus, the producers are executed after the selection step. 

In this part of the tutorial, we will write a producer which calculates the four lepton invariant mass.

